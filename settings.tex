\documentclass[12pt,reqno]{article}
%\usepackage[margin=1in]{geometry} 
\usepackage{amsmath}
\usepackage{tcolorbox}
\usepackage{amssymb}
\usepackage{amsthm}
\usepackage{mathrsfs}
\usepackage{lastpage}
\usepackage{fancyhdr}
\usepackage{accents}
\usepackage{tasks}
%\linespread{1.2}
%\parindent = 0pt
%\pagestyle{fancy}
\setlength{\parindent}{0pt}
\everymath{\displaystyle}
%new area
%\usepackage[utf8]{inputenc}
%\usepackage{CJKutf8}
%\xeCJKsetup{AutoFakeBold=true, AutoFakeSlant=true}

\usepackage{graphicx}

\usepackage{natbib}
\usepackage{tabularx}
\usepackage{multirow}
\usepackage{booktabs}
\usepackage{hyperref}
\usepackage{bm}
\usepackage{float}

\usepackage{multicol}





% we will modify sections, subsections and sub subsections
\RequirePackage{titlesec}
% Modification of section 
\titleformat{\section}[block]{\normalsize\bfseries\filcenter}{\thesection.}{.3cm}{} 


% modification of subsection and sub sub section
\titleformat{\subsection}[runin]{\bfseries}{ \thesubsection.}
{1mm}{}[.\quad]
\titleformat{\subsubsection}[runin]{\bfseries\itshape}{ \thesubsubsection.}
{1mm}{}[.\quad]

\newenvironment{solution}
  {\renewcommand\qedsymbol{$\blacksquare$}
  \begin{proof}[Solution]}
  {\end{proof}}
\renewcommand\qedsymbol{$\blacksquare$}

\newcommand{\ubar}[1]{\underaccent{\bar}{#1}}

%%%%%%%%%%%%%%%%%%%%%%%%%%%%%% Textclass specific LaTeX commands.
%\theoremstyle{plain}
\newtheorem{thm}{\textbf{Theorem}}[section]
\newtheorem{cor}{Corollary}[thm]
\newtheorem{lmma}[thm]{Lemma}
\newtheorem{prop}[thm]{Propersition}
\newtheorem*{defn}{\underline{Definition}}
\newtheorem*{ex*}{Example}
\newtheorem*{sol*}{Solution}
\newtheorem*{cor*}{Corollary}
\newtheorem*{thm*}{Theorem}
\newtheorem*{lmma*}{Lemma}
\newtheorem*{rmk*}{Remark}
\newtheorem*{pf*}{\underline{\textbf{Proof\ }}}

%%%%%%%%%%%%%%%%%%%%%%%%%%%%%% User specified LaTeX commands.
\renewcommand{\P}{\mathscr{P}}
\newcommand{\B}{\mathscr{B}}
\newcommand{\A}{\mathscr{A}}
\newcommand{\C}{\mathbb{C}}
\newcommand{\CC}{\mathscr{C}}
\newcommand{\R}{\mathbb{R}}
\newcommand{\Q}{\mathbb{Q}}
\newcommand{\Z}{\mathbb{Z}}
\newcommand{\N}{\mathbb{N}}
\newcommand{\X}{\mathcal{X}}
\newcommand{\T}{\mathscr{T}}
\newcommand{\arbuni}{\bigcup_{\alpha\in I}}
\newcommand{\finint}{\bigcap_{i=1}^n}
\newcommand{\Ua}{{\textsc{U}_\alpha}}
\newcommand{\Ui}{\textsc{U}_i}
\newcommand{\pair}[2]{\left( \,#1\,,\,#2\,\right) }
\newcommand{\dint}[2]{\int_{#1}^{#2}}
\newcommand{\sett}[1]{\left\{ \,#1 \,\right\}}
\newcommand{\linearcombination}[2]{#1_1#2_1+\cdots+#1_n#2_n}
\newcommand{\slinearcombination}[1]{#1_1+\cdots+#1_n}
\newcommand{\spann}[1]{\text{Span($#1$)}}
\newcommand{\sub}[1]{\text{sup}}
\newcommand{\inn}[1]{\left< #1 \right>}
\newcommand{\kernal}[1]{Ker(#1)}
\newcommand{\image}[1]{Im(#1)}
\newcommand{\redd}[1]{\textbf{\color{red}#1}}
%%%%%%%%%%%%%%%%%%%%%%%%%%%%%%
\title{Lecture Notes on Linear Algebra}
\author{CARAPAERO}
\date{}