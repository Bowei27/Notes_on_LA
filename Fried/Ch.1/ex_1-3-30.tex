\begin{solution}
			Suppose that we can write $v = x_1 + x_2 = y_1 + y_2$ with $x_1,y_1 \in W_1$ and $x_2,y_2 \in W_2$. Since $x_1 + x_2 = y_1 + y_2$ and $W_1,W_2$ are subsapces of a vector space $V$, so $x_1 - y_1 = x_2 - y_2 \in W_1 \cap W_2$. As $V =  W_1 \oplus W_2$, $x_1 - y_1 = x_2 - y_2 = \{0\}$, that is, $x_1 = y_1$ and $x_2 = y_2$. Hence it is a unique representation. Conversely, suppose that the condition holds. We claim that $V = W_1 + W_2$ and $W_1 \cap W_2 = \{0\}$. Since $W_1 + W_2$ is the smallest subspace of $V$, so $W_1 + W_2 \subseteq V$. By assumption, it is clear to get that $V \subseteq W_1 + W_2$. This proves $V = W_1 + W_2$. Since $W_1,W_2$ are subspaces of $V$, so $0 \in W_1$ and $0 \in W_2$. Let $v$ be a vector in $W_1 \cap W_2$. From the uniqueness of decomposition and $v = 0 + v = v + 0$, it implies that $v = 0$. This proves $W_1 \cap W_2 = \{0\}$. In conclusion, $V = W_1 \oplus W_2$.
		\end{solution}