\begin{proof}
	To prove $(a)$, we first show that $W_{1}+W_{2}$ is a subspace of $V$. Clearly, $0=0+0 \in W_{1}+W_{2}$. Let $x=x_{1}+x_{2}$, $y=y_{1}+y_{2}$, $c \in \mathbb{F}$

$cx+y=c(x_{1}+x_{2})+(y_{1}+y_{2})=(cx_{1}+y_{1})+(cx_{2}+y_{2}) \in W_{1}+W_{2}$

By corrolary 1.1.1, $W_{1}+W_{2}$ is a subspace of $V$.

Then we show that $W_{1}, W_{2} \subseteq W_{1}+W_{2}$. $\forall x \in W_{1} y \in W_{2}$, $x=x+0 \in W_{1}+W_{2}$, $y=0+y \in W_{1}+W_{2}$, we have $W_{1}, W_{2} \subseteq W_{1}+W_{2}$.

To prove $(b)$, Let $W$ be a subspace of $V$ contains both $W_{1}$ and $W_{2}$, then $\forall x \in W_{1} , y \in W_{2}$, $x+y \in W_{1}+W_{2}$. Thus $W_{1}+W_{2} \subseteq W$. 
\end{proof}